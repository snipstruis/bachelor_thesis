\section{Meta-Casanova}
  Meta-Casanova is a functional, declarative language.
  It allows for multiple implementations of functions called \textit{rules}.
  Rules may fail, in that case the next rule will be attempted.
  This will continue until a rule succeeds, or no rule matches in which case the program throws an exception.
  
  \subsection{Data}
  \texttt{Data} declarations declare a two-way \mbox{many-to-1} relation between types.
  This two-way relationships makes \texttt{Data} an \textit{alias}.

  \begin{code}
  Data "nil" -> list<'a>
  Data 'a -> "::" -> list<'a> -> list<'a>
  \end{code}

  In this example, the list type is declared with two constructors.
  They specify that a lists can be constructed in two ways: with \verb|nil| and with \verb|::| surrounded with a term of type \verb|'a|, and a term of type \verb|list<'a>|.

  Conversely, they also specify that an list can be destructed in two ways.
  The programmer will assert which destructor is expected, and the rule fails if the destructor does not match.
  An example of this is shown later, in subsection ``Funcs''.

  Additionally, constructors may be manipulated and partially applied like functions.
  This allows for greater flexibility at the cost that function and constructor names need to be unique in their namespace.

  \subsection{Polymorphism}
  Polymorphic data structures are supported with the \textbf{\texttt{is}} keyword.

  \begin{code}
  Data "error" -> string -> failableList<'a>
  failableList 'a is list<'a>
  \end{code}

  \noindent
  This means every constructor of the \verb|list| is also a valid constructor of \verb|failableList|, but not vice-versa.

  \subsection{Funcs}
  Func declarations specify a new function and its type.

  \begin{code}
  Func "length" -> list<'a> -> int
  \end{code}

  As with constructors, functions may be freely manipulated and partially applied, and have the restriction that their name must be unique in their namespace.

  \subsection{Rules}
  Meta-Casanova uses a syntax similar to that of natural deduction.
  For each Func declaration, there are one or more rules that define it.

  \begin{code}
  ---------------
  length nil -> 0

  length xs -> res
  ---------------------
  length x::xs -> 1+res
  \end{code}
  
  A rule is comprised of a line with below it on the left of the arrow the input, and on the right the output.
  The statements above the horizontal line are called \textit{premises}.
  They can be assignments like in the example above, or conditionals like \verb|a=b| or \verb|c<d|.

  We can now call the function \verb|length| with an example list:

  \begin{code}
    1::(2::nil) -> x
    length x    -> res
  \end{code}

  The first premise constructs a list called ``x'', and the second statement calls length with that list.
  The program will execute as follows:

  \begin{code}
  length 1::(2::nil)
      #\st{nil}#
      x::xs -> 1+(length 2::nil)
          #\st{nil}#
          x::xs -> 1+(length nil)
              nil -> 0
              #\st{x::xs}#
  \end{code}

  \noindent
  After which the function stops calling itself and starts accumulating the result on the way down.
  
  \begin{code}
         1+0 -> 1
     1+1 -> 2
  2
  \end{code}

  \noindent
  After which it tells us correctly that the length of the list 1::(2::nil) is indeed 2.
