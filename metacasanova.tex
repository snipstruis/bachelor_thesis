\section{Meta Casanova}\label{metacasanova}
It is necessary to understand a subset of Meta Casanova (MC) in order to understand the problem-space of the back-end.
Meta Casanova is a functional, declarative language.
This section will cover the subset of the language that is relevant for code generation.

\subsection{Data}\label{mcdata}
\texttt{Data} declarations declare a discriminated union~\cite{algebraic_datastructures}.
For example, we could define an inductive list as:

\begin{MC}
Data "nil" -> list<'a>
Data 'a -> "::" -> list<'a> -> list<'a>
\end{MC}

Which defines the same structure as this F\#-like pseudocode.

\begin{FS}
List<'a> = nil 
         | 'a :: List<'a>
\end{FS}

In this example, the list type is declared with two constructors.
They specify that a lists can be constructed in two ways: with \verb|nil| and with \verb|::| surrounded with a term of type \verb|'a|, and a term of type \verb|list<'a>|.

Conversely, they also specify that a list can be deconstructed in two ways.
The programmer will assert which deconstructor is expected, and the rule does not match if the deconstructor does not match.
An example of this is shown later.

Additionally, constructors may be manipulated and partially applied like functions.
This allows for greater flexibility, requiring only that function and constructor names be unique in their namespace.

%\subsection{Polymorphism}
%Polymorphic data structures are supported with the \textbf{\texttt{is}} keyword.
%
%\begin{code}
%Data "error" -> string -> failableList<'a>
%failableList 'a is list<'a>
%\end{code}
%
%\noindent
%This means every constructor of the \verb|list| is also a valid constructor of \verb|failableList|, but not vice-versa.

\subsection{Functions}
Function declarations specify a new function and its type.

\begin{MC}
Func "length" -> list<'a> -> int
\end{MC}

As with constructors, functions may be freely manipulated and partially applied, and have the restriction that their name must be unique in their namespace.

\subsection{Rules}
Meta Casanova uses a syntax similar to that of natural deduction.
%For each function declaration, there are one or more rules that define it.
It allows for multiple implementations of functions.
These implementations are called \textit{rules}.

Rules can fail to match.
If that happens, the rule will jump ahead to the next rule.
This this will continue until a rule succeeds, or no rule matches in which case the program throws a run-time exception.

\begin{MC}
---------------
length nil -> 0

length xs -> res
---------------------
length x::xs -> 1+res
\end{MC}

A rule is comprised of a line with below it on the left of the arrow the input, and on the right the output.
The statements above the horizontal line are called \textit{premises}\label{premises}.
They can be assignments like \verb|length xs -> res| in the example above, or conditionals like \verb|a==b| or \verb|c<d|.

In the case of assignments, they create a \textit{local identifier}.
These identifiers are local to the rule they appear in.
The input arguments of the rule are also local identifiers.

We can now call the function \verb|length| with an example list:

\begin{MC}
  1::(2::nil) -> x
  length x    -> res
\end{MC}

The first premise constructs a list called ``x'', and the second statement calls length with that list.
The program will execute as follows:

\begin{MC}[escapeinside=\#\#]
length 1::(2::nil)
    #\st{nil}#
    x::xs → 1+(length 2::nil)
        #\st{nil}#
        x::xs → 1+(length nil)
            nil → 0
            #\st{x::xs}#
\end{MC}

After which the function stops calling itself and starts accumulating the result on the way down.

\begin{MC}[gobble=2]
          1 ← 1+0
      2 ← 1+1
  2
\end{MC}

After which it tells us correctly that the length of the list 1::(2::nil) is indeed 2.

\subsection{Main}
Each program needs an entry point.
The entry point of an MC program is the main function.

\begin{MC}
    length 1::(2::nil) -> res
    -------------------------
    main -> res
\end{MC}

The results of the main function are printed on the console.
The previous program would therefore print \verb|2| on the screen.

