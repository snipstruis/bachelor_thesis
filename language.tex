\section{Output language}
The first decision had the most impact on the project, and was one that was difficult to change later on.

\textit{In what programming language does the code-generator produce its output?}

This may be different than the language the code-generator is written in.
The code-generator is written in F\#, like the rest of the compiler.
The reason we use F\#, rather than Meta-Casanova 2, is because of the debugging capabilities.

\subsection{Unmanaged}
Since speed was one of the requirements, I first looked at solutions with unmanaged parts.
Unmanaged code is code that is not interpreted by a runtime, but is instead executed directly.
%It is also much less restrictive than the .NET runtime because the .NET bytecode is strongly typed.\cite{ecma335}

The main advantage of unmanaged code is that the fast LLVM code generator can be used.
LLVM is a ``collection of modular and reusable compiler and toolchain technologies.''\cite{llvm}
Specifically, the LLVM optimizer is valuable.
It is used in the Clang, a C/C++ compiler on par with gcc, and with little effort can be use it to optimize our generated code.
This would mean we get all the optimisations of LLVM with relative ease.
It would however mean that we had to implement a garbage collector, as LLVM does not come with one.
 
.NET compatibility is also required, as explained in section \ref{whydotnet}.
There are a few systems that allow for managed and unmanaged code to communicate.
The most viable are P/invoke, C++/CLI interop, and a hosted runtime.

\subsubsection*{P/invoke}
Platform Invocation Services (P/invoke) allows managed code to call unmanaged functions that are implemented in a DLL.\cite{msdn_pinvoke}

This is the most common form of inter-op, and has great documentation.
However, there are two big disadvantages.

\begin{enumerate}
    \item .NET can only call native functions, not the other way around.
        This means that the bulk of the control flow happens inside .NET, minimizing the fast native code.
    \item Transfering data between .NET and native code has a high performance cost\cite{msdn_interop_performance}, since it has to be serialized.
        This overhead is so large that we expect it to negate any performance benefit from using native code.
\end{enumerate}

Because of this, P/Invoke was not chosen.

\subsubsection*{C++/CLI interop}
C++ for the Common Language Infrastructure (C++/CLI) is a programming language designed for interoperability with unmanaged code.\cite{msdn_c++cli}

While it seems like it does exactly what we need, it has portability issues.
The only C++/CLI compiler runs on windows and it only compiles for processors with the x86 architecture\cite{mono_c++cli}.
Besides that, non-typesafe operations (the main advantage of C++/CLI) are only allowed on windows.\cite{mono_c++cli}

This means C++/CLI is not cross-platform enough.

\subsubsection*{Hosted runtime}
It is possible to embed a .NET runtime inside a native program.
This would make it so the control flow takes place inside the native part.

This seems like the best solution out of the native hybrids.
However it still has two drawbacks.
The mono runtime has a different interface than the microsoft .NET api, leading to incompatible programs \cite{mono_embedding}.
The same large serialization overhead as P/Invoke is present\cite{msdn_hosted}.

\subsubsection*{conclusion}
None of the inter-op methods offer a satisfactory solution.
They all have downsides that outweigh the benefits.
It was decided to let go of the LLVM code-generation in favor of a more portable and reliable system.

\subsection{.NET}
Because of the problems with native hybrids, it was decided to choose a .NET language.
Stability is a big advantage because everything happends inside the .NET runtime.
This has a higher chance of working on non-native platforms than the hybrid solutions.

\subsubsection*{F\#}
F\# would be a natural choice, since the compiler is written in it.
However, it is quite slow\cite{fsharp_slow} and resists the imperative style of the generated code.
% maybe look for Mohammeds compiler written in F#?

\subsubsection*{C\#}
C\# has more advantages.
It is the most popular .NET language\cite{Meyerovich}, so the compiler gets the most attention by Microsoft.
It is also easy to debug, as it has the most mature debugging tools.

\subsubsection*{CIL}
CIL (Common Intermediate Language) is the bytecode that all the languages are compiled to.
Since it is typed, it has the same restrictions as C\#.[source]
As a result, it makes debugging and verification harder, with little to no gain.

\subsection{Conclusion}
The debugability together with a lot of control make C\# the best choice in this case.

