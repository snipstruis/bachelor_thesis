\section{Output language}
The first decision had the most impact on the project, and was one that was difficult to change later on.

\textit{In what programming language does the code-generator produce its output?}

This may be different than the language the code-generator is written in.
The code-generator is written in F\#, like the rest of the compiler.


\subsection{Unmanaged}
Since speed was one of the requirements, I first looked at solutions with unmanaged parts.
Unmanaged code is code that is not interpreted by a runtime, but is instead executed directly.
It is also much less restrictive than the .NET runtime because the .NET bytecode is strongly typed.\cite{ecma335}

Another advantage of unmanaged code is that the fast LLVM C/C++ code generator can be used.
--- explanation LLVM here ---
This would mean we get all the optimisations of LLVM with very little effort.

.NET compatibility is also required.
There are a few systems that allow for managed and unmanaged code to communicate.
The most viable are P/invoke, C++/CLI interop, and a hosted runtime.

\subsubsection*{P/invoke}
Platform Invocation Services (P/invoke) allows managed code to call unmanaged functions that are implemented in a DLL.\cite{msdn_pinvoke}

large overhead because of marshalling \cite{msdn_interop_performance}

\subsubsection*{C++/CLI interop}
C++ for the Common Language Infrastructure (C++/CLI) is a programming language designed for interoperability with unmanaged code.%\cite{msdn_c++cli}

compiler windows-only\cite{mono_c++cli}

not typesafe(not allowed in safe-mode) pure CIL is windows only\cite{mono_c++cli}

only on x86\cite{mono_c++cli}

\subsubsection*{Hosted runtime}
large overhead, no fast jit.

\subsubsection*{conclusion}
none are good enough.


\subsection{.NET}
Because of the problems go with .NET.
A big advantage is stability because everything happends inside the .NET runtime.
This has a higher chance of working on non-native platforms than the hybrid unmanaged solutions.

\subsubsection*{F\#}
F\# would be a natural choice, since the compiler is written in it.
However, it is quite slow[source] and resists the imperative style of the generated code.

\subsubsection*{C\#}
C\# has more advantages.
It is the most popular .NET language, and the compiler gets the most attention by Microsoft.
It is also easy to debug, as it has the most mature debug tools.

\subsubsection*{CIL}
CIL (Common Intermediate Language) is the bytecode that all the languages are compiled to.
Since it is typed, it has the same restrictions as C\#.[source]
So it makes debugging and verification harder, with little to no gain.

\subsection{Conclusion}
The debugability together with a lot of controll make C\# the best choice in this case.
