\subsection{Choice of output language}
The first research question had the most impact on the project, and was one that was difficult to change later on.

\textit{In what programming language should the code-generator produce its output?}

This may be different than the language the code-generator is written in.
The code-generator is written in F\#, like the rest of the compiler.
The reason we use F\#, rather than Meta Casanova 2, is because Meta Casanova 2 lacks tool support, such as descriptive error messages and debuggers.

\subsubsection{Unmanaged languages}
Since speed was one of the requirements, I first looked at solutions with unmanaged parts.
Unmanaged code is code that is not interpreted by a runtime, but is instead executed directly.
%It is also much less restrictive than the .NET runtime because the .NET bytecode is strongly typed~\cite{ecma335}.

The main advantage of unmanaged code is that the fast LLVM code generator can be used.
LLVM is a ``collection of modular and reusable compiler and toolchain technologies.''~\cite{llvm}
Specifically, the LLVM optimizer is valuable.
It is used in the Clang, a C/C++ compiler on par with GCC, and with little effort can be use it to optimize our generated code.
This would mean we get all the optimisations of LLVM with relative ease.
It would however mean that we had to implement a garbage collector, as LLVM does not come with one~\cite{llvmgc}.
 
.NET compatibility is also required, as explained in section~\ref{whydotnet}.
There are a few systems that allow for managed and unmanaged code to communicate.
The most viable are P/invoke, C++/CLI inter-op, and a hosted runtime.

\paragraph{P/invoke}
Platform Invocation Services allows managed code to call unmanaged functions that are implemented in a DLL~\cite{msdn_pinvoke}.

This is the most common form of inter-op, and has great documentation.
However, there are two big disadvantages.

\begin{enumerate}
    \item .NET can only call native functions, not the other way around.
        This means that the bulk of the control flow happens inside .NET, minimizing the fast native code.
    \item Transferring data between .NET and native code has a high performance cost~\cite{msdn_interop_performance}, since it has to be serialized.
        This overhead is so large that we expect it to negate any performance benefit from using native code.
\end{enumerate}

Because of this, P/Invoke was not chosen.

\paragraph{C++/CLI}
C++ for the Common Language Infrastructure is a programming language designed for interoperability with unmanaged code~\cite{ecma372}.

While it seems like it does exactly what we need, it has portability issues.
The only C++/CLI compiler runs on windows and it only compiles for processors with the x86 architecture~\cite{mono_c++cli}.
Besides that, non-type-safe operations (the main advantage of C++/CLI) are only allowed on windows~\cite{mono_c++cli}.

This means C++/CLI is not cross-platform enough.

\paragraph{Hosted runtime}
It is possible to embed a .NET runtime inside a native program.
This would make it so the control flow takes place inside the native part.

This seems like the best solution out of the native hybrids.
However it still has two drawbacks.
The mono runtime has a different interface than the Microsoft .NET API, leading to incompatible programs ~\cite{mono_embedding}.
The same large serialization overhead as P/Invoke is present~\cite{msdn_hosted}.

\subsubsection{.NET languages}
None of the inter-op methods offer a satisfactory solution.
They all have downsides that outweigh the benefits.
It was decided to let go of the LLVM code-generation in favor of a more portable and reliable system.

Stability is a big advantage because everything happens inside the .NET runtime.
This has a higher chance of working on non-native platforms than the hybrid solutions.

\paragraph{F\#} is a functional/declarative language in the .NET family~\cite{smith}.
It would be a natural choice, since the compiler is written in it.
It has the advantage of supporting tail-calls, which is unsupported by C\#,
  and since F\# already supported algebraic data types, it seemed like a viable solution.
%The programmer has also less control of the program execution, as F\# is a more higher-level language~\cite{fsharp}
%A version of the code generator was made that used F\#, but it proved too cumbersome since there was no mechanism to simply skip rules if they did not match.

% maybe look for Mohammeds compiler written in F#?

\paragraph{C\#}
is an imperative, object-oriented language~\cite{ecma334}.
It is the most popular .NET language~\cite{Meyerovich}, so the compiler gets the most attention by Microsoft.
It is also easy to debug, as it has the most mature debugging tools.
C\# too seemed like a viable option.

%When implementing the code-generator in this style, it was possible to write the code-generator so that the output-code is straight-line code.
%This is opposed to F\#, where I had to generate a tree-structure as an output.
%This greatly improved debugability and simplicity of implementation.

\paragraph{CIL}
(Common Intermediate Language) is the bytecode that all the languages are compiled to.
Since it is typed, it has the same restrictions as C\#~\cite{ecma335}.
As a result, it makes debugging and verification harder, with little to no gain.
It also omits the optimizations of the C\# compiler, such as dead-code elimination\footnote{see section~\ref{recommendations}}.

\subsubsection{Conclusion}
The result of the research was that C\# and F\# were both viable.
To choose an output language, I built a code model for each language.

The F\# code model mainly involved switching off indentation-based scoping and using the verbose syntax.
Scoping was implemented by making each rule a numbered function that returned an \verb|Option<>|.
The numbered rules were then called and appended using the \verb|>>=| of the \verb|Option| monad.

While this model worked, it was cumbersome and slow.
The code was hard to inspect, since there was no indentation.
It was also relatively slow because:

\begin{enumerate}
\item It had to wrap each return value in a \verb|Option|. 
    This is particularly costly for value-types, as they have to be boxed and unboxed each time the value is used~\cite{cisternino}.
\item It performs monadic operations for each rule attempt.
    These generic functions do type resolution at run-time, each time they are called~\cite{cisternino}.
\pagebreak
\item The numbered functions were not inlined, preventing any cross-rule optimization\footnote{information obtained by inspecting generated CIL code of the Microsoft F\# compiler \texttt{fsc.exe} version 1.0 with \texttt{+optimize}}.
\end{enumerate}

The C\# code model proved far easier to generate and inspect, as it had braces and local scopes.
It is the model that is now used. The C\# code model is explained in detail in section~\ref{codegen}.
