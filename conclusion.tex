\section{Results}
The result is a working, reliable, performant back-end, with interpreter, validator and embedded debugger.

Unfortunately, since the front-end was incomplete, it is not possible to compile source files.
It \textit{is} however possible to write the front-end interface by hand.

Two such test programs were written.

\subsection{list length test}
The first program defined a list datastructure and a program to compute its length.
This was used since it uses each basic instruction at least once, as well as matching.
It is equivalent to the following MC code:

\begin{MC}
Data int -> "::" -> List -> List
Data "nil" -> List

Func "length" -> List -> int

---------------
length nil -> 0

lengtht xs -> res
---------------------
length x::xs -> res+1

--------------------------------
main -> length (1::2::3::4::nil)
\end{MC}

Which when executed prints the following on screen.

\begin{lstlisting}
4
\end{lstlisting}

This program can also be debugged with the embedded debugger.

\subsection{XNA test}

The second test program was to test the .Net functionality.
It consists of a simple program that modifies XNA datastructures, specifically the \verb|Vector2|.


\section{Conclusions}
--- summary here ---
\subsection{Recommendations}
--- propose optimization ---
\subsection{Reflection}
It went well.
