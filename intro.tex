\section{Intro}
--- abstract here ---
This thesis also aims to be useful to the future developers of the MC compiler.

\subsection{Rationale}
Writing compilers is a difficult task.

Compilers are complex programs that have to operate on a wide range of inputs.
Since compilers have such a large input-space, the chance of a bug hiding somewhere is substantial. 
But compilers also have to be bug-free, since every program can only be as bug-free as its compiler.

Abstractions can help in this regard.
The limits of which were observed when implementing the compiler for the Casanova language in F\#.

Since F\# has no higher-order type operators, it is not possible to write monad-transformers.
This hampers modularity and resulted in a lot of boilerplate code.

The Casanova compiler written in F\# was 0000 lines long, and became unmaintainable.
After a rewrite in MC it was 000 lines[Maggiore].

This project is about the back-end of the bootstrap compiler voor Meta-Casanova 3, written in F\#.
The back-end is responsible for generating an executable after receiving the type-checked information from the front-end.

%\begin{itemize}
%    \item Casanova needed for multiplayer games.
%    \item Compiler proved too difficult in F\#
%    \item Stronger abstractions needed
%    \item MC has those abstractions
%    \item the compiler required a back-end
%    \item Back-end is responsible for creating executable
%\end{itemize}

\subsection{Organization}

\subsection{Goals}
\begin{itemize}
    \item decided on an output language
    \item designed front-end interface
    \item designed Intermediate Representation
    \item made Validator
    \item made Code generator with code mangler
    \item made interpreter
\end{itemize}

\subsection{Requirements}
\begin{itemize}
    \item Correct
    \item .NET Compatible
    \item Multi-platform
    \item fast
\end{itemize}
