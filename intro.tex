\section{Intro}
--- abstract here ---
This thesis also aims to be useful to the developers of the MC compiler.

\subsection{Rationale}
Writing compilers is a difficult task.

Compilers are complex programs that have to operate on a wide range of inputs.
Since compilers have such a large input-space, the chance of a bug hiding somewhere is substantial. 
But compilers also have to be bug-free, since every program can only be as bug-free as its compiler.

Abstractions can help in this regard.
The limits of which were observed when implementing the compiler for the Casanova language in F\#.

Since F\# has no higher-order type operators, it is not possible to write monad-transformers.

The F\# compiler was 0000 lines long, and became unmaintainable.
After a rewrite in MC it was 000 lines[source].

\begin{itemize}
    \item Casanova needed for multiplayer games.
    \item Compiler proved too difficult in F\#
    \item Stronger abstractions needed
    \item MC has those abstractions
    \item the compiler required a back-end
    \item Back-end is responsible for creating executable
\end{itemize}

\subsection{Organization}

\subsection{Goals}
\begin{itemize}
    \item code generator
    \item embedded debugger
    \item code validator
\end{itemize}

\subsection{Requirements}
\begin{itemize}
    \item Correct
    \item .NET Compatible
    \item Multi-platform
    \item fast
\end{itemize}
