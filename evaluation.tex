\section{Evaluation}\label{evaluation}
In this section I will show that I have the competences associated with computer science according to Rotterdam University of applied sciences.

\subsubsection{Administering} 
I used Visual Studio 2015 on windows, even though my primary development environment is on linux.
This was primarily so that the project kept working well with the other members of the research group.
During this, I initiated the move away from SVN to Git, to make us more productive.
To make it easy for everyone on the research group, I made sure Visual Studio had a Git plugin.

\subsubsection{Analyzing}
I have split the back-end problem up in managable, testable chunks.
I researched compiler architectures like the LLVM framework, which inspired the early normalization (called cannonicallization in LLVM).
It also inspired the SSA form and the inclusion of type-information.

\subsubsection{Advising}
This thesis also serves as a documentation of the back-end to the stakeholders.
It contains a `future work' section that contain advice on how to improve the back-end.

\subsubsection{Designing}
The parts of the back-end are modular and communicate with each other through well-defined interfaces.
This made it easy to respond to changes in the language, as changes in one part have no effect on the other parts.
It was also easy to test and debug the seperate parts because they only depend on their explicit inputs.

\subsubsection{Realizing}
I managed to to realize a working compiler back-end within the allocated time.
It comes with this thesis as documentation so the future developers of the back-end can easily get started.

