\section{Mangler}
The mangler is responsible for generating a unique C\# identifier for every instance of an MC identifier.
The mangler is designed to be simple, and produce readable output.
Readable output makes it easy to verify both the mangler and the generated code.

There are two kinds of identifier: global identifiers and local identifiers.
Global identifiers have a fully-qualified name with type information, where as local identifiers only have the simple name.

\subsection{C\# identifiers}
Since there are more valid MC identifier names than C\# identifier names, some characters have to be escaped.

Valid C\# identifiers are \verb|[_A-Za-z][_A-Za-z0-9]*| [source].
The only valid non-alphanumeric character is an underscore, so that is used to escape with.

The first iteration of the code mangler just replaced all non-numeric characters with an underscore followed with the two-digit hexadecimal number.
This generated correct identifiers but was very unreadable, \verb|>>=| would translate to \verb|_3E_3E_3D|.
To remedy this, every ascii symbol gets a readable label.

\begin{tabular}{ll|ll|ll}
\verb0!0 & \verb0_bang0  & \verb0-0 & \verb0_dash0  & \verb0=0 & \verb0_equal0 \\
\verb0#0 & \verb0_hash0  & \verb0.0 & \verb0_dot0   & \verb0?0 & \verb0_quest0 \\
\verb0$0 & \verb0_cash0  & \verb0/0 & \verb0_slash0 & \verb0@0 & \verb0_at0    \\ %$
\verb0%0 & \verb0_perc0  & \verb0\0 & \verb0_back0  & \verb0^0 & \verb0_caret0 \\
\verb0&0 & \verb0_amp0   & \verb0:0 & \verb0_colon0 & \verb0_0 & \verb0_under0 \\
\verb0'0 & \verb0_prime0 & \verb0;0 & \verb0_semi0  & \verb0`0 & \verb0_tick0  \\
\verb0*0 & \verb0_amp0   & \verb0<0 & \verb0_less0  & \verb0|0 & \verb0_pipe0  \\
\verb0+0 & \verb0_plus0  & \verb0>0 & \verb0_great0 & \verb0~0 & \verb0_tilde0 \\
\verb0,0 & \verb0_comma0 \\
\end{tabular}

\subsection{reserved words}
C\# allows reserved words to be used as valid identifiers if prefixed with an \verb|@|.[source]

\subsection{types}
Global identifiers need type information embedded in the name since the name alone does uniquely identify it (see thingy).
Types can be recursive (see types), so the system for embedding types must be able to represent tree structures.
The simplest way to represent tree structures with text is the way Lisp did it: S-expressions.

The reason this differs from the syntax of types is partly historical.

\begin{tabular}{ll}
\textbf{type}          & \textbf{mangled} \\
\verb|array<int,3>|    & \verb|array_Tint_T3| \\
\verb|list<list<int>>| & \verb|list_Llist_Tint_R| \\
\end{tabular}

The symbols used are: \verb|_T| as seperator, \verb|_L| for the left parenthesis, and \verb|_R| for the right parenthesis.


