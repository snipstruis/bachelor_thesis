\subsection{Mangler}
The Mangler could be seen as part of the code-gen, but the decisions are intresting enough to get its own research question.

\textit{How to generate the identifiers so they comply with the output language?}

The mangler is responsible for generating a unique C\# identifier for every instance of an MC identifier.
The mangler is designed to be simple, and produce readable output.
Readable output makes it easy to verify both the mangler and the generated code.

There are two kinds of identifier: global identifiers and local identifiers.
Global identifiers have a fully-qualified name with type information, where as local identifiers only have the simple name.

\subsubsection{C\# identifiers}
Since there are more valid MC identifier names than C\# identifier names, some characters have to be escaped.

Valid C\# identifiers must start with an alphabetic character or an underscore and the trailing characters must be alphanumeric or underscore\footnote{regex: \texttt{[\_A-Za-z][\_A-Za-z0-9]*}}~\cite{msdn_identifiers}.
The only valid non-alphanumeric character is an underscore, so using it to escape with was a logical choice.

The first iteration of the code mangler just replaced all non-numeric characters with an underscore followed with the two-digit hexadecimal number.
This generated correct identifiers but was very unreadable, \verb|>>=| would translate to \verb|_3E_3E_3D|.
To remedy this, every ASCII symbol gets a readable label.

\begin{tabular}{ll|ll|ll}
\verb0!0 & \verb0_bang0  & \verb0-0 & \verb0_dash0  & \verb0=0 & \verb0_equal0 \\
\verb0#0 & \verb0_hash0  & \verb0.0 & \verb0_dot0   & \verb0?0 & \verb0_quest0 \\
\verb0$0 & \verb0_cash0  & \verb0/0 & \verb0_slash0 & \verb0@0 & \verb0_at0    \\ %$
\verb0%0 & \verb0_perc0  & \verb0\0 & \verb0_back0  & \verb0^0 & \verb0_caret0 \\
\verb0&0 & \verb0_amp0   & \verb0:0 & \verb0_colon0 & \verb0_0 & \verb0_under0 \\
\verb0'0 & \verb0_prime0 & \verb0;0 & \verb0_semi0  & \verb0`0 & \verb0_tick0  \\
\verb0*0 & \verb0_amp0   & \verb0<0 & \verb0_less0  & \verb0|0 & \verb0_pipe0  \\
\verb0+0 & \verb0_plus0  & \verb0>0 & \verb0_great0 & \verb0~0 & \verb0_tilde0 \\
\verb0,0 & \verb0_comma0 \\
\end{tabular}

\subsubsection{Reserved words}
C\# allows reserved words to be used as valid identifiers if prefixed with an `\verb|@|'~\cite{msdn_identifiers}.

\subsubsection{Types}
Global identifiers need type information embedded in the name since the name alone does uniquely identify it.
Types can be recursive\footnote{see section~\ref{mcdata}}, so the system for embedding types must be able to represent tree structures.
We use the same syntax as the front-end but with \verb|_S| as seperator, \verb|_L| for the left angle bracket and \verb|_R| for the right angle bracket.

{\footnotesize
\begin{tabular}{ll}
\textbf{\normalsize type}          & \textbf{\normalsize mangled} \\
\verb|array<int,3>|    & \verb|array_Lint_S3_R| \\
\verb|list<list<int>>| & \verb|list_Llist_Lint_R_R| \\
\end{tabular}
}

\subsubsection{Evolution}
The first iteration of the mangler just numbered every identifier.
While this was a simple system to generate identifiers with, it was absolutely impossible to inspect the resulting code.
Most of the mangler is the result of a desire for readable, inspectible output code.

