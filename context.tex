\section{Context}\label{context}
The graduation assignment is carried out at Kenniscentrum Creating 010.
\textit{Kenniscentrum Creating 010 is a transdisciplinary design-inclusive Research Center enabling citizens, students and creative industry making the future of Rotterdam}~\cite{creating2016home}.

The assignment is carried out within a research group that is building a new programming language.
The new programming language is called \textit{Casanova}.

\subsection{Research group}
The research group is creating the Casanova language.
The members of the research group are 
  Francesco di Giacomo\footnote{\label{venice}Universita' Ca' Foscari, Venezia}, 
  Mohamed Abbadi\footnoteref{venice}, 
  Agostino Cortesi\footnoteref{venice}, 
  Giuseppe Maggiore\footnote{Hogeschool Rotterdam} and 
  Pieter Spronck\footnote{Tilburg University}.

Within the research group is our research team, tasked with the design and implementation of Meta Casanova.
The research team is supervised by Giuseppe Maggiore and comprises of three students.
Louis van der Burg, responsible for developing the Meta Casanova language,
Jarno Holstein, responsible for the front-end of the Meta Casanova compiler,
and Douwe van Gijn, responsible for the back-end of the Meta Casanova compiler.

\subsection{Motive}\label{motive}
Kenniscentrum is interested in innovative technologies.
Innovative technologies like virtual reality and video games are the fields our research group is researching.

In order to ease the development of virtual reality and video games, the Casanova language was developed.
The Casanova language is the subject of the PhD thesis of Francesco.

The complex nature of the Casanova language lead to a complex compiler.
To simplify the development of Casanova, the language Meta Casanova was developed.

