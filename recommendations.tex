\section{Recommendations}
The back-end is in a functional state, but it could use more features.
Most of the features are optimizations.
Optimizations generally have a relatively small effect, but often take up a lot of development time.

\subsubsection{Inlining}
The most important optimization is inlining~\cite{inlining}.
Inlining is the process of replacing a function call with the function body.
While this saves a function call, the greatest benefit is that it enables other optimizations.

When the function body is copied in the larger context of the callsite, some input values may be identified as compile-time constants, enabeling a whole array of optimizations. 

Inlining is not always desireable.
If a large function is called from many different places, the size of the program increases, increasing the cache-misses on the instruction cache, reducing performance.

The choice for inlining a call should consider:
\begin{enumerate}
    \item If the function recurses in any way, even indirect recursion.
        This makes inlining impossible.
    \item The size of the function.
        The smaller the function, the greater the inlining benefit.
    \item The ammount of times the function is called.
        If the function is only called once, inlining has no disadvantages.
        For each additional time, the size of the program increases.
\end{enumerate}

\subsubsection{Tail call optimization}
--- use gotos ---

\subsubsection{Constant folding}
--- C\# does this a bit (see \url{http://stackoverflow.com/a/5669035}) ---

