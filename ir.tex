\subsection{Intermediate Representation}\label{ir}
While the intermediate representation (IR) of the functions is part of the interface, it is complex enough to have its own research question.

\textit{What should the intermediate representation of the functions be?}

Each rule contains a list of premises\footnote{see section~\ref{premises}}.
These premises represent the executable code in each rule.

To minimize the number of representations of the same program, all compound premises are split into multiple premises that do only one operation each.
This process is called \textit{normalization}\label{normalization}.

The instruction set exists in two parts: the base instructions and the .NET extensions.

\subsubsection{Base instructions}
The instruction set was designed to minimize the number of representations of the same program.
This happens to coincide with a small orthogonal instruction set.

The instruction set is in \textit{static single assignment} (SSA) form.
This means the local identifiers are constant and can not be redefined~\cite{leestatic}.

Base instructions fall in one of two groups.
The first maps a global identifier to a local identifier.
These are the \textit{Literal} and \textit{Closure} instructions.
The second operates on local identifiers.
The \textit{Conditional}, \textit{Deconstructor}, \textit{Application} and \textit{Call} instructions belong to this group.

\begin{description}
\item[Literal] (\verb|42 -> x|) assigns a string-, boolean-, integer- or floating-point literal to a local identifier.
\item[Conditional] (\verb|x < y|) asserts that a comparison between local identifiers is true.
    The comparisons can be \verb|<|,\verb|<=|,\verb|==|,\verb|>=|,\verb|>| or \verb|!=|.
    If the assertion does not match, the rule does not match and the next rule in the function is attempted.
\item[Deconstructor] (\verb|lst -> x::xs|) disassembles a local identifier constructed by a data declaration.
\item[Closure] (\verb|(+) -> add|) assigns a closure of a global function to a local identifier.
    The closure can hold a function, lambda or data-constructor.
\item[Application] (\verb|add a -> inc|) applies a local identifier to a closure in another local identifier.
\item[Call] (\verb|inc b -> c|) applies a local identifier and calls the closure.
    All closures need to be called eventually to be useful.
    The exceptions are data-constructors. 
    They do not have to be called as they insert their elements in the data structure as they are applied.
\end{description}

\subsubsection{.NET extensions}
A separate set of instructions are needed to inter-operate with .NET.
This is because unlike MC, .NET objects are mutable, and the functions can be overloaded on the number and types of arguments.

\begin{tabular}{ll}
    \textbf{instruction} & \textbf{MC example}\\
    call & \begin{lstlisting}
        date.toString format -> str
    \end{lstlisting}\\

    static call & \begin{lstlisting}
        System.DateTime.parse str -> date
    \end{lstlisting}\\

    get & \begin{lstlisting}
        date.DayOfWeek -> day
    \end{lstlisting}\\

    static get & \begin{lstlisting}
        System.DateTime.Today -> day
    \end{lstlisting}\\

    set & \begin{lstlisting}
        hr -> foo.hour
    \end{lstlisting}\\

    static set & \begin{lstlisting}
        hr -> System.DateTime.LocalTime
    \end{lstlisting}\\ 
\end{tabular}

\subsubsection{Evolution}
The IR changed a lot during development.
It got simpler with each iteration.

\paragraph{Existing IR} It was briefly considered to use an existing intermediate representation, like CIL or LLVM-IR.
However, it would mean over 100 instructions and the front-end would do most of the work.
It would also mean the front-end needed its own code generator to generate the CIL instructions.

\paragraph{Call} Call did not used to apply an argument, but it caused inconsistencies in the type-checker.
There would be not difference in the type of the uncalled closure and the called closure, resulting in an extra bit of information being required with the type.
This caused special-cases all over the codebase, so it was decided to make application take an argument, like in lambda-calculus.

\paragraph{Application} Application used to also take the position of the argument that was applied.
This was because the back-end did not care in what order the closures were applied.
But since the MC language only allows for in-order closure application, the decision was made to make the position of the argument implicit to limit the program representations.

\paragraph{Comparisons} Comparisons could first only take a boolean local identifier.
It was changed to a predefined set of comparisons because of two reasons.
Firstly, it makes the language-agnostic base instructions depend on .NET Booleans.
Secondly, by restricting the inputs to only a predefined set of comparisons, we restrict the number of representations for the same program.

