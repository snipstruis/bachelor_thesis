\section{Code generator}
The codegen is the heart of the back end.
It is responsible for generating the C\# code.

\subsection{Function declarations}
Function

\begin{code}
class #\textit{<function name>}# {
    #\textit{<function arguments>}#
    public #\textit{<return type>}# 
    run(#\textit{<last argument>}#) {
        {
            #\textit{<rule 1 implementation>}#
            return #\textit{<local>}#;
        }
      skip1:
        {
            #\textit{<rule 2 implementation>}#
            return #\textit{<local>}#;
        }
      skip2:
        #\vdots#
        {
            #\textit{<rule n implementation>}#
            return #\textit{<local>}#;
        }
      skip#\textit{n}#:
        throw new #\textit{<exception>}#;
    }
};
\end{code}

\subsection{Data declarations}
Data declarations are implemented with inheritance.
The declared type is represented by an empty baseclass and all the constructors inherit from it.

It is easy to see the pattern with an example.

\begin{code}
Data string -> "," -> int -> string * int

Data "Left"  -> string -> string | float
Data "Right" -> float  -> string | float
\end{code}

Transforms into this.

\begin{code}
class _star {};
class _comma { string _arg0; int _arg1;}

class _pipe {};
class _Left :_pipe {string _arg0;};
class _Right:_pipe {float  _arg0;};
\end{code}

\subsection{Rules}

Each rule defines its own name for each input argument.
These names do not have to be the same, for example:

\begin{code}
    Func "evenOrOdd" -> int -> string
    
    a%2 = 1
    -----------------
    evenOrOdd a -> "odd!"

    b%2 = 0
    ------------------
    evenOrOdd b -> "even!"
\end{code}

Of course, by the time the code has arived by the codegen, it would already have been normalized.
So the rules actually look more like this:

\begin{code}
    (%) -> _tmp0         #\textit{(closure)}#
    _tmp0 a -> _tmp1     #\textit{(application)}#
    2 -> _tmp2           #\textit{(literal)}#
    _tmp1 _tmp2 -> _tmp3 #\textit{(call)}#
    0 -> _tmp4           #\textit{(literal)}#
    tmp4 = tmp0          #\textit{(conditional)}#
    "even" -> _tmp5      #\textit{(literal)}#
    --------------------
    evenOrOdd a -> _tmp5
\end{code}

\begin{code}
    (%) -> _tmp0         #\textit{(closure)}#
    _tmp0 a -> _tmp1     #\textit{(application)}#
    2 -> _tmp2           #\textit{(literal)}#
    _tmp1 _tmp2 -> _tmp3 #\textit{(call)}#
    1 -> _tmp4           #\textit{(literal)}#
    tmp4 = tmp0          #\textit{(conditional)}#
    "odd" -> _tmp5       #\textit{(literal)}#
    --------------------
    evenOrOdd a -> _tmp5
\end{code}

The first job of the rule is to translate the input arguments to their name and return the output.

\begin{code}
    {
        var a = _arg0; 
        ...
        return _tmp5;
    }
    _skip0:
    {
        var b = _arg0;
        ...
        return _tmp5;
    }
    _skip1:
\end{code}

Then each instruction is generated.

\begin{code}
    {
        var a = _arg0; 
        // closure
        var _tmp0 = new _plus(); 
        // application
        var _tmp1 = add;
        _tmp1._arg0 = a;
        // literal
        var _tmp2 = 2;
        // call
        var _tmp3 = _tmp1.run(_tmp2);
        // literal     
        var _tmp4 = 1;
        // conditional
        if(!(_tmp3=_tmp4)){goto _skip0;}
        // literal
        "odd!" -> _tmp5;
        return _tmp5;
    }
    _skip0:
    {
        var b = _arg0;
        ...
        return _tmp5;
    }
    _skip1:
\end{code}

\end{multicols}
\minipage{\linewidth}
figure 1: \textit{an overview of instruction generation.}\\
\begin{tabular}{lll}
    \hline
    \textbf{instruction} & \textbf{MC} & \textbf{C\#}\\

    \hline
    literal & \begin{lstlisting}
        42 -> x
    \end{lstlisting} & \begin{lstlisting}
        var x = 42;
    \end{lstlisting} \\

    \hline
    conditional & \begin{lstlisting}
        x > 40  
    \end{lstlisting} & \begin{lstlisting}
        if(!(x>40)){goto skip0;}
    \end{lstlisting} \\

    \hline
    deconstructor & \begin{lstlisting}
        lst -> x::xs 
    \end{lstlisting} & \begin{lstlisting}
        var _tmp0 = lst as _colon_colon;
        if(_tmp0==null){goto _skip0;}
        var x  = _tmp0._arg0;
        var xs = _tmp0._arg1;
    \end{lstlisting} \\

    \hline
    closure & \begin{lstlisting}
        (+) -> add
    \end{lstlisting} & \begin{lstlisting}
        var add = new _plus();
    \end{lstlisting} \\

    \hline
    application & \begin{lstlisting}
        add a -> inc
    \end{lstlisting} & \begin{lstlisting}
        var inc = add;
        inc._arg0 = a;
    \end{lstlisting} \\

    \hline
    call & \begin{lstlisting}
        inc b -> c
    \end{lstlisting} & \begin{lstlisting}
        var c = inc.run(b);
    \end{lstlisting} \\

    \hline
    \textbf{.NET instr.} & \textbf{MC} & \textbf{C\#}\\

%    \hline
%    constructor & \begin{lstlisting}
%        System.DateTime d m y -> date
%    \end{lstlisting} & \begin{lstlisting}
%        var date = System.DateTime(d,m,y);
%    \end{lstlisting} \\

    \hline
    call & \begin{lstlisting}
        date.toString format -> str
    \end{lstlisting} & \begin{lstlisting}
        var str = date.toString(format);
    \end{lstlisting} \\

    \hline
    static call & \begin{lstlisting}
        System.DateTime.parse str -> date
    \end{lstlisting} & \begin{lstlisting}
        var date = System.DateTime.parse(str);
    \end{lstlisting} \\

    \hline
    get & \begin{lstlisting}
        date.DayOfWeek -> day
    \end{lstlisting} & \begin{lstlisting}
        var day = date.DayOfWeek;
    \end{lstlisting} \\

    \hline
    static get & \begin{lstlisting}
        date.DayOfWeek -> day
    \end{lstlisting} & \begin{lstlisting}
        var day = date.DayOfWeek;
    \end{lstlisting} \\

    \hline
    set & \begin{lstlisting}
        hr -> System.DateTime.hour
    \end{lstlisting} & \begin{lstlisting}
        System.DateTime.hour = hr;
    \end{lstlisting} \\

    \hline
    static set & \begin{lstlisting}
        hr -> System.DateTime.hour
    \end{lstlisting} & \begin{lstlisting}
        System.DateTime.hour = hr;
    \end{lstlisting} \\
    \hline
\end{tabular}
\endminipage
%\pagebreak
\begin{multicols}{2}
 % double width and page-breaking

\subsection{Evolution}
C\# unions only work with value-types.

