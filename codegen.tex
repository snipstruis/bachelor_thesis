\subsection{Code generator} \label{codegen}
The fourth research question gets at the heart of the back-end.

\textit{How does the intermediate representation map to the output language?}

The code generator is in many ways the heart of the back-end, as it is responsible for generating the C\# code.

\subsubsection{Functions}
Every function was implemented as a closure.
In C\# this means a class with a public field for each function argument and a \verb|_run| function that takes the last argument and executes the function.

\begin{CS}[escapeinside=\#\#]
class #\textit{<function name>}# {
    #\textit{<function arguments>}#
    public #\textit{<return type>}# 
    _run(#\textit{<last argument>}#) {
        {
            #\textit{<rule 1 implementation>}#
            return #\textit{<local>}#;
        }
      _skip1:
        {
            #\textit{<rule 2 implementation>}#
            return #\textit{<local>}#;
        }
      _skip2:
        #\vdots#
        {
            #\textit{<rule n implementation>}#
            return #\textit{<local>}#;
        }
      skip#\textit{n}#:
        throw new #\textit{<exception>}#;
    }
};
\end{CS}

The \verb|_run| function opens a local scope followed by a goto-label for each rule in the function.
This allows rules to easily jump ahead to the label when they do not match\footnote{see subsection \textit{Rules}.}

\subsubsection{Data declarations}
Data declarations are implemented with inheritance.
The declared type is represented by an empty base class and all the constructors inherit from it.

This is a pretty straight-forward transformation.

\begin{MC}
Data string -> "," -> int -> Tuple

Data "Left"  -> string -> Union
Data "Right" -> float  -> Union
\end{MC}

The above MC code transforms into the following C\# code.

\begin{CS}
class Tuple{}
class _comma { string _arg0; int _arg1;}

class Union{}
class Left :Union {string _arg0;}
class Right:Union {float  _arg0;}
\end{CS}

The types \verb|Tuple| and \verb|Union| can now be easily be deconstructed.
When a premise deconstructs a datatype, it asserts that a type is constructed by a specific constructor.
This is done by simply casting the base-class to a subclass, and checking if the cast succeeded.
If the cast failed, the rule does not match and the rule is skipped.

\subsubsection{Rules}\label{codegen_rules}

Each rule defines its own name for each input argument.
These names do not have to be the same, for example:

\begin{MC}
    Func "evenOrOdd" -> int -> string
    
    a%2 = 1
    -----------------
    evenOrOdd a -> "odd!"

    b%2 = 0
    ------------------
    evenOrOdd b -> "even!"
\end{MC}

Of course, by the time the code has reached by the code generator, it would already have been normalized.
So the rules actually look more like this:

\begin{MC}[escapeinside=\#\#]
    (%) -> _tmp0         #\textit{(closure)}#
    _tmp0 a -> _tmp1     #\textit{(application)}#
    2 -> _tmp2           #\textit{(literal)}#
    _tmp1 _tmp2 -> _tmp3 #\textit{(call)}#
    1 -> _tmp4           #\textit{(literal)}#
    _tmp4 = _tmp0        #\textit{(conditional)}#
    "odd!" -> _tmp5      #\textit{(literal)}#
    --------------------
    evenOrOdd a -> _tmp5
\end{MC}

\begin{MC}[escapeinside=\#\#]
    (%) -> _tmp0         #\textit{(closure)}#
    _tmp0 b -> _tmp1     #\textit{(application)}#
    2 -> _tmp2           #\textit{(literal)}#
    _tmp1 _tmp2 -> _tmp3 #\textit{(call)}#
    0 -> _tmp4           #\textit{(literal)}#
    _tmp4 = _tmp0        #\textit{(conditional)}#
    "even!" -> _tmp5     #\textit{(literal)}#
    --------------------
    evenOrOdd b -> _tmp5
\end{MC}

The first job of the rule is to translate the input arguments to their name and return the output.

\begin{CS}
    {
        var a = _arg0; 
        ...
        return _tmp5;
    }
    _skip0:
    {
        var b = _arg0;
        ...
        return _tmp5;
    }
    _skip1:
\end{CS}

Then each instruction is generated.

\begin{CS}[escapeinside=\#\#]
    {
        var a = _arg0; 
        // closure
        var _tmp0 = new _plus(); 
        // application
        var _tmp1 = add;
        _tmp1._arg0 = a;
        // literal
        var _tmp2 = 2;
        // call
        var _tmp3 = _tmp1.run(_tmp2);
        // literal     
        var _tmp4 = 1;
        // conditional
        if(!(_tmp3=_tmp4)){goto _skip0;}
        // literal
        "odd!" -> _tmp5;
        return _tmp5;
    }
    _skip0:
    {
        var b = _arg0;
        #\textit{<omitted for brevity>}#
        return _tmp5;
    }
    _skip1:
\end{CS}

See the figure 1 on the next page for an overview of instruction generation.

\subsubsection{Main function}
The main function is the entry point of the program.
When the program is run, the main function is called and the result is printed.

The body of the rule is generated like any rule in its own function.
This improves readability by separating the main function specific code from the general rule code.

\begin{CS}[escapeinside=\#\#]
class _main{
    static #\textit{<return type>}# _body(){
        #\textit{<main rule>}#
    }
    static void Main(){
        #\textit{<debugger initialization>}#
        System.Console.WriteLine(
            System.String.Format("{0}", 
                                 _body()));
    }
}
\end{CS}

\end{multicols}
\minipage{\linewidth}
figure 1: \textit{an overview of instruction generation.}\\
\begin{tabular}{lll}
    \hline
    \textbf{instruction} & \textbf{MC} & \textbf{C\#}\\

    \hline
    literal & \begin{lstlisting}
        42 -> x
    \end{lstlisting} & \begin{lstlisting}
        var x = 42;
    \end{lstlisting} \\

    \hline
    conditional & \begin{lstlisting}
        x > 40  
    \end{lstlisting} & \begin{lstlisting}
        if(!(x>40)){goto skip0;}
    \end{lstlisting} \\

    \hline
    deconstructor & \begin{lstlisting}
        lst -> x::xs 
    \end{lstlisting} & \begin{lstlisting}
        var _tmp0 = lst as _colon_colon;
        if(_tmp0==null){goto _skip0;}
        var x  = _tmp0._arg0;
        var xs = _tmp0._arg1;
    \end{lstlisting} \\

    \hline
    closure & \begin{lstlisting}
        (+) -> add
    \end{lstlisting} & \begin{lstlisting}
        var add = new _plus();
    \end{lstlisting} \\

    \hline
    application & \begin{lstlisting}
        add a -> inc
    \end{lstlisting} & \begin{lstlisting}
        var inc = add;
        inc._arg0 = a;
    \end{lstlisting} \\

    \hline
    call & \begin{lstlisting}
        inc b -> c
    \end{lstlisting} & \begin{lstlisting}
        var c = inc.run(b);
    \end{lstlisting} \\

    \hline
    \textbf{.NET instr.} & \textbf{MC} & \textbf{C\#}\\

%    \hline
%    constructor & \begin{lstlisting}
%        System.DateTime d m y -> date
%    \end{lstlisting} & \begin{lstlisting}
%        var date = System.DateTime(d,m,y);
%    \end{lstlisting} \\

    \hline
    call & \begin{lstlisting}
        date.toString format -> str
    \end{lstlisting} & \begin{lstlisting}
        var str = date.toString(format);
    \end{lstlisting} \\

    \hline
    static call & \begin{lstlisting}
        System.DateTime.parse str -> date
    \end{lstlisting} & \begin{lstlisting}
        var date = System.DateTime.parse(str);
    \end{lstlisting} \\

    \hline
    get & \begin{lstlisting}
        date.DayOfWeek -> day
    \end{lstlisting} & \begin{lstlisting}
        var day = date.DayOfWeek;
    \end{lstlisting} \\

    \hline
    static get & \begin{lstlisting}
        date.DayOfWeek -> day
    \end{lstlisting} & \begin{lstlisting}
        var day = date.DayOfWeek;
    \end{lstlisting} \\

    \hline
    set & \begin{lstlisting}
        hr -> System.DateTime.hour
    \end{lstlisting} & \begin{lstlisting}
        System.DateTime.hour = hr;
    \end{lstlisting} \\

    \hline
    static set & \begin{lstlisting}
        hr -> System.DateTime.hour
    \end{lstlisting} & \begin{lstlisting}
        System.DateTime.hour = hr;
    \end{lstlisting} \\
    \hline
\end{tabular}
\endminipage
%\pagebreak
\begin{multicols}{2}
 % double width and page-breaking

\subsubsection{Evolution}
The code generator changed little because the code model was developed in the early stages to research if C\# was viable.
There were only two instances where the code generator changed during development.

Before using inheritance, the plan was to use overlapping memory like C unions. 
Using \verb|System.|\verb|Runtime.|\verb|InteropServices|, it was possible to set the specific offset of struct members.
By overlapping fields in memory, we could achieve the same effect as C union.
While this was multiplatform and worked well, it only worked with structs.
This was a major limitation, because structs can only hold \textit{value types} and the only value types are other structs and \textit{simple types} like integers, floats, and booleans.
This was a problem since most of the .Net objects are classes, and only a few of the .Net objects are value-types.

The second change was a simplification that made it much easier to have breakpoints for the embedded debugger.
The code generator used to produce a nested structure before we used \verb|goto|.
The if-statements were nested for the rest of the rule, for example:

\begin{CS}
    if(x>10){
        var foo = 20;    
        if(x<100){
            return foo;
        }
    }
\end{CS}

As opposed to the \verb|goto| model:

\begin{CS}
    if(!(x>10)){goto _skip0;}
    var foo = 20;    
    if(!(x<100)){goto _skip0;}
    return foo;
\end{CS}

It took a lot more work to traverse this tree-like structure in order to insert breakpoints at the appropriate places, particularly because these breakpoints added a another if-statement and thus another layer of nesting\footnote{see section~\ref{debugger}}.
The size of the code shrunk a lot when the decision was made to use \verb|goto|s, and the code became a lot more readable.

